\documentclass[12 pt]{article}
\usepackage{hyperref, fancyhdr, setspace, enumerate, amsmath,
  lastpage, amssymb}
\usepackage[margin=1 in]{geometry}
\allowdisplaybreaks
%\usepackage[dvipsnames]{xcolor}   %May be necessary if you want to color links
\hypersetup{
	%colorlinks=true, %set true if you want colored links
	linktoc=all,     %set to all if you want both sections and subsections linked
	linkcolor=black,  %choose some color if you want links to stand out
}
\usepackage{graphicx}
\graphicspath{{Images/}}
\author{Julian Lore}
\date{Last updated: \today}
\title{COMP 424: Artificial Intelligence}
\pagestyle{fancy}
\lhead{COMP 424}
\chead{\leftmark}
\rhead{Julian Lore}
\cfoot{Page \thepage \ of \pageref{LastPage}}
\newcommand{\tab}[1]{\hspace{.2\textwidth}\rlap{#1}}
\begin{document}
	\onehalfspacing
	\maketitle
	Notes from Jackie Cheung's Winter 2018 lectures and slides
        (which are based on slides by Joelle Pineau).
	\tableofcontents
        \section{Introduction to AI 01/08/18}
        \href{http://cs.mcgill.ca/~jcheung/teaching/winter-2018/comp424/index.html}{Class
          website}. Material will be posted on myCourses.
        \paragraph{Some of the relevant terms we'll be hearing in this
          course}
        \begin{itemize}
        \item Planning
        \item Reasoning
        \item Search (very important)
        \item Modeling
        \item Learning (big part that we will address during the
          second half of the semester)
        \item Decision-making
        \item Perception
        \item Language
        \item Knowledge
        \end{itemize}
        \paragraph{Course Topics}
        \begin{itemize}
        \item Search
        \item Game playing
        \item Logical reasoning
        \item Classical planning
        \item Probabilistic reasoning
        \item Learning probabilistic models
          \\\noindent\rule{\textwidth}{0.5pt}
        \item Post midterm:
        \item Reasoning with utilities
        \item Sequential reasoning and decision-making
        \item Learning complex sequential decisions
        \item Applications
        \end{itemize}
        \subsection{Biological Intelligence}
            Before we talk about artificial intelligence, let's talk about
            biological intelligence.
            \paragraph{Sensory processing} used to perceive things
            \begin{itemize}
            \item Visual cortex
            \item Auditory cortex
            \item Somatosensory cortex
            \end{itemize}
            \paragraph{Motor cortex} then used to act on things
            \paragraph{Cognitive functions}
            \begin{itemize}
            \item Memory
            \item Reasoning
            \item Executive control
            \item Learning
            \item Language
            \end{itemize}
            Biological intelligence is a mix of general-purpose and
            special-purpose algorithms.
            \paragraph{General-purpose}
            \begin{itemize}
            \item Memory formation, updating, retrieval
            \item Learning new tasks
            \end{itemize}
            \paragraph{Special-purpose}
            \begin{itemize}
            \item Recognizing visual patterns
            \item Recognizing sounds
            \item Learning language
            \end{itemize}
            All are integrated seamlessly and we are not always aware
            of what is going on.
        \subsection{Artificial Intelligence}
            What is AI? There are several answers.
            \begin{itemize}
            \item Modeling human cognition using computers
            \item Studying problems that others can't solve, i.e. some
              complex optimization problems like scheduling and
              minimizing conflicts. Perhaps this is too hard to solve
              for people with pen and paper.
            \item Game playing, machine learning, data mining, speech
              recognition, computer vision, web agents, robots
            \item Medical diagnosis, fraud detection, genome analysis,
              object identification, space shuttle scheduling,
              information retrieval
            \end{itemize}
            Solving AI may be the solution to all problems, as we
            would just write an AI to solve it.
            \paragraph{Working definition} Developing models and
            algorithms that can produce rational behaviors in response
            to incoming stimulus and information.
            \paragraph{Mapping to human intelligence} We can map human
            intelligence to certain aspects of AI.
            \begin{itemize}
            \item Visual cortex relates to computer vision
            \item Auditory cortex, signal/speech processing
            \item Somatosensory cortex, haptics
            \item Motor cortex, robotics
            \item Memory, knowledge representation
            \item Reasoning, search/inference
            \item Executive control, planning/decision-making
            \item Learning, model learning
            \item Language, language understanding
            \end{itemize}
            Will not be focusing on the sensory part in this course,
            mainly the core cognitive functions.
            \paragraph{Different Goals of AI}
            \begin{itemize}
            \item Thinking like a human
            \item Thinking rationally
            \item Acting like a human
            \item Acting rationally (the focus of this course, as we
              may note with the working definition)
            \end{itemize}
            We don't have good answers on how to model thinking, so it
            is very hard to model the first two points. People have
            proposed several models, like a logic based model, but
            others disagree. It is not easy to observe people
            thinking, but it is easy to observe people's
            actions. Acting humanly would be nice, but it has several
            problems.
            \subparagraph{Acting Human}
            It's hard to define acting human but also
            perhaps we may not always want to act like a human.
            \paragraph{}Alan Turing made the \textbf{Turing test} with
            a human judge and one human and one AI agent on the other
            side. The goal is for the judge to not be able to tell
            which one is the human and which one isn't. It is still
            not very easy to evaluate AI systems to determine whether
            or not they ``act human'' or not. Also, people may make
            different judgments for the same thing, i.e. it can be
            shown in a positive way and the person might decide
            differently in comparison to it being shown in a negative
            way, yet the choice is still numerically the same. 
            \subparagraph{Acting Rationally}
            It's about doing the ``right'' thing, although this has to
            be decided by the creator. Essentially, the AI must
            maximize goal achievement with its available information
            and resources. Doesn't always require thinking, but often
            does.
            \\ The AI agent does actions influencing the world and
            then gets information from its task and environment
            (perception). Might perceive before acting. In two player
            games, the AI thinks based on what the opponent
            played. But what if we exploit that and play based off
            what the opponent will play if we play some move.
        \paragraph{Rational Agents}
            This course is about making rational agents. Agents
            perceive and act. Our goal is to learn a function that
            maps percept histories to actions. $$f:P^h \to A$$
            Might not always act the same way based off the percept
            history, i.e. a game may want to implement some
            randomness. This function does not mean the agent is
            deterministic. A rational agent implements this function
            to maximize performance (measured by goal achievement,
            resource consumption and more).
            \\ We have resource constraints (like time, space, energy,
            bandwidth) that make perfect rationality impossible. So
            instead our objective is to find the best function for
            given information and resources.
        \subsection{History of AI}
        \begin{itemize}
        \item ENIAC was the first super-computer created in 1946
        \item Early work done in 50s, perceptron and agent that can
          play checkers
        \item Dartmouth Conference in 1956 to propose studying AI
        \end{itemize}
            \paragraph{Early Ambitions}
                Make programs with similar intelligence as people,
                such that they can prove theorems, play chess and have
                a conversation. Logical reasoning was heavily used and
                learning was important. Didn't want a system that can
                only do one thing, like play checkers, wanted a very
                general purpose machine, but did not workout. Since
                people were too optimistic about AI, it lead to many
                disappointments and the field grew slower.
            \paragraph{Recently}
                More math heavy, lots of probability theory, decision
                theory and statistics. Aimed for specific problems
                instead of general problems. Some may argue that this
                helps in the short term, but not the long term. More
                and more people are thinking about how we can go back
                to thinking about putting all the sub-fields together,
                like how AI originally was aiming for. AI is
                now a collection of sub-fields.
            \paragraph{Chess Playing (1997)} Perception consists of
            advanced features of the board, each spot on the board and
            how certain positions are better than others. Actions
            consist of making a move, there are concrete actions in
            chess, valid and invalid moves. Reasoning, searching for a
            move and evaluating possible board positions, what kind of
            state it will leave you in. Computer resources are
            important, the more resources you have, the more you can
            look ahead.
            \paragraph{Poker Playing (2008)} Perception consists of
            the features of game and the actions consist of moves. 
            \paragraph{Jeopardy (2011)} Jeopardy is about
            trivia, sometimes worded in a convoluted way so some
            rational thinking is required. Perception would be clue
            you get and the category,
            action would be deciding whether or not you hit the buzzer
            and what your response is. Reasoning would be searching
            through database of knowledge, looking for relevant information.
            \paragraph{Atari (2015)} Learning, gets a signal that
            signifies whether or not the action the machine is
            performing is good or not and the agent will tend to do
            things that are better.
            \paragraph{Self-Play for Go (2017)} Original version of
            AlphaGo learned by original plays by human experts and
            additional training reinforcements to get good, but this
            version just played by itself and got good, did not need
            to look at human games or need human help.
            \noindent\rule{\textwidth}{0.5pt}
            So we have seen great results in AI, but games are much
            more constrained than other things, such as natural
            language processing, the amount of actions may be
            infinite, say if we don't bound the length of the
            sentence, while in games there are only a finite number of
            actions. There is also no good reward/good behavior, like
            there is in games (i.e. you're winning). In some ways, our
            AI systems are better than us (i.e. games), but in many
            other ways they aren't, some of the domains we want to
            apply AI to are much more complex.
            \paragraph{Stock Market} AI agents are now much more
            involved in the stock market, they tell us when to trade
            given the rates and the news. Perception consists of rates
            and news, actions are trades and reasoning is putting all
            this information together. These automated trading agents
            are so quick now that it actually matters how close they
            are to the stock/server.
            \paragraph{Medical Diagnosis (1992)} Perception: symptoms,
            test results, actions: suggest tests, make diagnosis,
            reasoning: bayesian inference, machine learning,
            Monte-Carlo. Might have to make decisions with missing
            information, what's the best choice?
            \paragraph{City Driving (2014)} Google cars, almost no
            accidents.
            \paragraph{Amazon Echo (2016)} Consumer-grade AI for less
            than \$200 US that is useful to us.
            \paragraph{AI and the Web} There is a lot of information
            available on the web in order for agents to learn, like
            search engines and social networking websites. So many
            interactions that we must now use large data processing algorithms.
\end{document}
