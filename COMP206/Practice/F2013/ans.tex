\documentclass[12 pt]{article}
\usepackage{hyperref}
\usepackage{fancyhdr}
\usepackage{setspace}
\usepackage{enumerate}
\usepackage{amsmath}
\usepackage{lastpage}
\usepackage{mathtools,float}
\usepackage{tabularx}
\usepackage{amssymb}
\usepackage{listings}
\usepackage[dvipsnames]{xcolor}
\usepackage[margin=1 in]{geometry}
\allowdisplaybreaks
\usepackage{graphicx}
\pagestyle{fancy}
\lhead{COMP 206}
\chead{\leftmark}
\rhead{Julian Lore}
\cfoot{Page \thepage \ of \pageref{LastPage}}
\newcommand{\tab}[1]{\hspace{.2\textwidth}\rlap{#1}}
\begin{document}
\onehalfspacing
Time taken (before correction and edits): $1:09$
\paragraph{Question 1: HTML and CGI}~
\lstinputlisting[language=html]{Q1.html}
\paragraph{Question 2: GNU}
\begin{enumerate}[(A)]
\item ~
  \begin{enumerate}[(a)]
  \item ~
\begin{lstlisting}[language=bash]
CC = gcc # Sets a variable to gcc, for compiler
CFLAGS = -g -Wall # Var for compiler flags

count: countwords.o counter.o scanner.o
# If the 3 o files exist, make the count file
# Or if one of the o files is newer than the count file
    $(CC) $(CFLAGS) -o count countwords.o counter.o scanner.o
    # Compiles all the o files as count, with all the warnings
    # and inserts timestops for GProf

countwords.o: countwords.c scanner.h counter.h
# Make countwords.o if source and corresponding headers
# are present
    $(CC) $(CFLAGS) -c countwords.c
    # Makes countwords.o, displays all errors, GProf

counter.o: counter.c counter.h
# Make counter.o if source and headers are there
    $(CC) $(CFLAGS) -c counter.c
    # counter.o, same as above

scanner.o: scanner.c scanner.h
    $(CC) $(CFLAGS) -c scanner.c
    # Same as counter.o

clean:
    $(RM) count *.o *~
    # Removes count and all .o files and files ending in ~
\end{lstlisting}
  \item Scanner.o depends on scanner.h and scanner.c, counter.o depends on counter.h and counter.c, countwords.o depends on countwords.c scanner.h counter.h and count depends on all .o files.
    \item Huge box that is just the count program. Within it are smaller rectangles, one for countwords.o, one for counter.o and one for scanner.o
  \end{enumerate}
\item ~
\begin{lstlisting}[language=bash]
(gdb) break main
(gdb) run
(gdb) step
(gdb) step
(gdb) ... # Until in makeit()
(gdb) step
(gdb) step
(gdb) ... # Eventually program will crash
(gdb) where # Find out where you are when it crashed
(gdb) backtrace
\end{lstlisting}
\end{enumerate}
\paragraph{Question 3: C Programming}~
\begin{lstlisting}[language=c]
// Function header, returns int, takes int pointer
int foo(int *pn)
{
    int c, s; // Makes 2 ints
    // Loops while user inputs a space, CR or tab
    while ((c=getch() == ' ' || c == '\n' || c == '\t'));

    s = 1; 
    if (c == '+' || c == '-') // c is + or -
    {
    // If c is +, s=1, else s=-1
    s = (c=='+') ? 1 : -1;
    c = getch(); // Get a char again
    }

    // Make contents of pointer 0
    // Loop while c is inbetween 0 and 9 (ASCII)
    // Get character at end of each loop
    for(*pn=0; c >= '0' && c <= '9'; c=getch())
        // 10 times pointers val + # that c stores
        *pn = 10 * *pn + c – '0';
    
    *pn *= s; // Make pn the same sign as s
    if (c != EOF) ungetch(c); // If not end of file
    return c; // return 2nd to last char obtained
}
\end{lstlisting}
This program gets a sign from the user, then keeps getting numbers from the user,
summing them up and multiplying by 10 everytime a new number comes along. This will be available to the user since it's a pointer,
but the actual function returns the 2nd to last obtained char.
\paragraph{Question 4: C Programming}~
\lstinputlisting[language=c]{Q4.c}
\paragraph{Question 5: Python Programming}~
\lstinputlisting[language=python]{Q5.py}
\end{document}
%%% Local Variables:
%%% mode: latex
%%% TeX-master: t
%%% End:
