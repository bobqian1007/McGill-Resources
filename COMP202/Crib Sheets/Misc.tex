~\\ \color{Black}
\colorbox{Yellow}{Misc} Var with same name can be declared 2x in same method (i.e. 2 for loops)
\\ Index out of bounds $\rightarrow$ run time error , printing in a non-void method will still print
\\ You can make two arrays of diff size equal to another, just changes address it points at, int[] a= new int[5]; int[] b = new int[10]; a=b; will work!
\\ \colorbox{Red}{Remember negation distributes!} !(a || b)$\rightarrow$!a\&\&!b
\\ Bytecode is result of compiling
\\ Constant doesn't need memory
\\ a \&\& b and b\&\& a don't eval to same result, if first is false, says error, maybe first gives compiler error instead
\\ Compiler error $\rightarrow$ everything looks fine, runtime error $\rightarrow$ logic doesn't work
\\ Can compile java file without main method
\\ Default val of int[] is 0 in each pos
\\ System.out.println(..); is a non static method (called on out)
\\ Can overload methods, make 2 methods with same name, but diff input, will call suitable one
\\ Cannot overload methods with diff return type, compiler error
\\ Every java prog has at least 1 var declared
\\ A non-static method can access a static var in same class
\\ ArrayList cannot store *different* types of values
\\ Can assign a line from reader to String[], can also use name.split(String) to split around matches of that String
\\ ''$\backslash$t`` is a tab character for String
\\ Trying to access uninitiated String[] args is an ArrayIndexOutOfBoundsException
\\ Out of bounds is a runtime error $\rightarrow$ will still compile!
\\ Writing a diff method that is static vs non-static is not considered overloading
\\ Can initialize double with Double d = new Double(29.95);