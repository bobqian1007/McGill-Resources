~\\ \color{Fuchsia}
10. Exceptions
\\ Impossible to exec $\rightarrow$ exception or runtime error
\\ Info from exceptions: 
\\ Exception in thread \textquotedblleft main\textquotedblright \colorbox{Cyan}{java.util.InputMismatchException}
\\ \phantom{abc}\textcolor{Orange}{at java.util.Scanner.throwFor (\colorbox{Brown}{Scanner.java:819})}
\\ \textcolor{Cyan}{Type of exception}
\\ \textcolor{Orange}{Stack trace, which methods called method that crashed prog}
\\ \textcolor{Brown}{Line number and file that exception occured in}
\\ \colorbox{Green}{Array exceptions}
\\ ArrayIndexOutOfBounds $\rightarrow$ try to access invalid index, exception gives index number that caused it
\\NullPointerException (access properties of null var)
\\ \colorbox{Green}{Throwing} Some commands can't be executed given certain input, instead of using ifs and whatnot to make output null, should throw, problem will become hard to find if not
\\ Immediately generate an error by using throw, rather than hide
\\ if(stuff==null || \ldots)\{throw new IllegalArgumentException(\textquotedblleft Invalid input \textquotedblright)\}
\\ Other kinds of exceptions: DivideByZero, NumberFormatException (String to number, etc.)
\\With throw, method will give value or error
\\ If you want to process the job, use try/catch
\\ try{ some stuff;} catch(IllegalArgumentException e)\{Happens if that type of exception happens, if no error, skip this, if different type of exception, pass to caller\}
\\ Diff types of Exceptions, hierarchy 
\\ Can use catch(Exception e)\{ To catch all exceptions\} | \colorbox{Red}{Dangerous though!} Can hide all bugs
\\ Can put multiple catch blocks after, first one that matches will be executed
\\ Each method in method chain can catch exception: main()$\rightarrow$a()$\rightarrow$b()$\rightarrow$c()
\\ If c gens exception, can try/catch in b. If not, can try/catch in a, if not, then main. If none catch, then prog crashes
\\ Can also use try/catch/finally, finally will occur no matter what, good for: removing dupe code, clean up before crash/return
\\ Can even do try/finally
\\ \colorbox{Green}{3 types of errors} Compile time, run time, bug(incld infinite loop) $\rightarrow$ compile gives most info, run a bit, bug none
\\ Throw used to give runtime error rather than bug
\\ Exception is an object, catch(Exception e) declares var e of type Exception
\\ Can create one via new Exception(\textquotedblleft info \textquotedblright), make own exception type for particular kind of prob in your code