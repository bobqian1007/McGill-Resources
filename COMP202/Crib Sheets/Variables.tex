~\\ \color{Brown}
2. Variables
\\ Place in mem reserv to store val. Java: give variables \colorbox{yellow}{name} and \colorbox{yellow}{type}
\\ Why? Store partial results, generalize code, easier to understand
\\ Names can contain \colorbox{yellow}{letters, numbers, and \_} | Names should explain purpose | Convention: degreesInFahrenheit
\\ Can also define your own types of data
%
\\ \colorbox{Green}{Creating var} Need 1) Var's type \& 2) Var's name $\rightarrow$ called \colorbox{yellow}{declaring} a variable. i.e. int fahrenheitNumber;
\\ Declaring variables important, comp needs to know how much mem to allocate \& makes easier for you
\\ Setting value of variable: variable = expression; | i.e. fahrenheit=212; |express will be eval before assignd
\\ Set val of undecl var = compiler error
\\ ``String literal". takes as string, if you put var in quotes won't take var
%
\\\colorbox{Red}{!!} Vars made in 1 block \colorbox{red}{not rel} to vars in other block!
%
\\ \colorbox{Green}{Cmd line args} public static void main (String[] args)
\\ args is a var, set by comp when prog starts
\\ Type of args is String[] (String array)
\\ $1^{st}$ String accessed by args[0], $2^{nd}$ String by args[1] \ldots
\\ run Test $\underbrace{100}_{\textnormal{args[0]}}$ $\underbrace{50}_{\textnormal{args[1]}}$
\\ If u parse args that aren't type parsed $\rightarrow$ runtime error