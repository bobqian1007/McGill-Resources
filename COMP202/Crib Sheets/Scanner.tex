~\\ \color{Brown}
11. Scanner
\\ Need to \colorbox{Yellow}{import library}, import java.util.Scanner;
\\ Object, ref type (like array \& exception)
\\ Declare new: Scanner reader = new Scanner(System.in); System.in to get input from user
\\ Read vals from user: scannerName.nextInt(); .nextDouble(); .next(); (until space) .nextBoolean();
\\ \colorbox{Green}{Reading from file} 
\\ Scanner fileRdr = new Scanner(new File(''foo.txt'')); foo in same folder as .class file | need to \colorbox{Yellow}{import} java.io
\\ Need to deal with exception from this! Need to \colorbox{Yellow}{import} java.io.IOException;
\\ try\{... new File..\} catch(IOException e)\{..\}
\\ Can \colorbox{Yellow}{just import java.io.*}
\\ \colorbox{Green}{Mandatory Exception Handling}
\\ FileNoteFoundException is a ''checked exception``, mandatory to catch or add to method header: public Scanner(File f) throws FileNotFoundException\{\}
\\ Unchecked exceptions are the usual, like ArrayIndexOutOfBoundsException, etc.
\\ To write to a file, use FileWriter and BufferedWriter: (Buffered to write temp to RAM, faster but not necessary)
\\ Put this inside try/catch or throw in header
\\ FileWriter fw= new FileWriter(''file.txt``); if file.txt exists, will overwrite. If not, creates
\\ BufferedWriter bw = new BufferedWriter(fw);
\\ bw.write(StringName); bw.newLine(); bw.write(3+'' ``);
\\ bw.close(); close it, will get written to fw
\\ fw.close(); always close at end